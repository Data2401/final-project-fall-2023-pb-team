% Options for packages loaded elsewhere
\PassOptionsToPackage{unicode}{hyperref}
\PassOptionsToPackage{hyphens}{url}
%
\documentclass[
  ignorenonframetext,
]{beamer}
\usepackage{pgfpages}
\setbeamertemplate{caption}[numbered]
\setbeamertemplate{caption label separator}{: }
\setbeamercolor{caption name}{fg=normal text.fg}
\beamertemplatenavigationsymbolsempty
% Prevent slide breaks in the middle of a paragraph
\widowpenalties 1 10000
\raggedbottom
\setbeamertemplate{part page}{
  \centering
  \begin{beamercolorbox}[sep=16pt,center]{part title}
    \usebeamerfont{part title}\insertpart\par
  \end{beamercolorbox}
}
\setbeamertemplate{section page}{
  \centering
  \begin{beamercolorbox}[sep=12pt,center]{part title}
    \usebeamerfont{section title}\insertsection\par
  \end{beamercolorbox}
}
\setbeamertemplate{subsection page}{
  \centering
  \begin{beamercolorbox}[sep=8pt,center]{part title}
    \usebeamerfont{subsection title}\insertsubsection\par
  \end{beamercolorbox}
}
\AtBeginPart{
  \frame{\partpage}
}
\AtBeginSection{
  \ifbibliography
  \else
    \frame{\sectionpage}
  \fi
}
\AtBeginSubsection{
  \frame{\subsectionpage}
}
\usepackage{amsmath,amssymb}
\usepackage{iftex}
\ifPDFTeX
  \usepackage[T1]{fontenc}
  \usepackage[utf8]{inputenc}
  \usepackage{textcomp} % provide euro and other symbols
\else % if luatex or xetex
  \usepackage{unicode-math} % this also loads fontspec
  \defaultfontfeatures{Scale=MatchLowercase}
  \defaultfontfeatures[\rmfamily]{Ligatures=TeX,Scale=1}
\fi
\usepackage{lmodern}
\ifPDFTeX\else
  % xetex/luatex font selection
\fi
% Use upquote if available, for straight quotes in verbatim environments
\IfFileExists{upquote.sty}{\usepackage{upquote}}{}
\IfFileExists{microtype.sty}{% use microtype if available
  \usepackage[]{microtype}
  \UseMicrotypeSet[protrusion]{basicmath} % disable protrusion for tt fonts
}{}
\makeatletter
\@ifundefined{KOMAClassName}{% if non-KOMA class
  \IfFileExists{parskip.sty}{%
    \usepackage{parskip}
  }{% else
    \setlength{\parindent}{0pt}
    \setlength{\parskip}{6pt plus 2pt minus 1pt}}
}{% if KOMA class
  \KOMAoptions{parskip=half}}
\makeatother
\usepackage{xcolor}
\newif\ifbibliography
\usepackage{color}
\usepackage{fancyvrb}
\newcommand{\VerbBar}{|}
\newcommand{\VERB}{\Verb[commandchars=\\\{\}]}
\DefineVerbatimEnvironment{Highlighting}{Verbatim}{commandchars=\\\{\}}
% Add ',fontsize=\small' for more characters per line
\usepackage{framed}
\definecolor{shadecolor}{RGB}{248,248,248}
\newenvironment{Shaded}{\begin{snugshade}}{\end{snugshade}}
\newcommand{\AlertTok}[1]{\textcolor[rgb]{0.94,0.16,0.16}{#1}}
\newcommand{\AnnotationTok}[1]{\textcolor[rgb]{0.56,0.35,0.01}{\textbf{\textit{#1}}}}
\newcommand{\AttributeTok}[1]{\textcolor[rgb]{0.13,0.29,0.53}{#1}}
\newcommand{\BaseNTok}[1]{\textcolor[rgb]{0.00,0.00,0.81}{#1}}
\newcommand{\BuiltInTok}[1]{#1}
\newcommand{\CharTok}[1]{\textcolor[rgb]{0.31,0.60,0.02}{#1}}
\newcommand{\CommentTok}[1]{\textcolor[rgb]{0.56,0.35,0.01}{\textit{#1}}}
\newcommand{\CommentVarTok}[1]{\textcolor[rgb]{0.56,0.35,0.01}{\textbf{\textit{#1}}}}
\newcommand{\ConstantTok}[1]{\textcolor[rgb]{0.56,0.35,0.01}{#1}}
\newcommand{\ControlFlowTok}[1]{\textcolor[rgb]{0.13,0.29,0.53}{\textbf{#1}}}
\newcommand{\DataTypeTok}[1]{\textcolor[rgb]{0.13,0.29,0.53}{#1}}
\newcommand{\DecValTok}[1]{\textcolor[rgb]{0.00,0.00,0.81}{#1}}
\newcommand{\DocumentationTok}[1]{\textcolor[rgb]{0.56,0.35,0.01}{\textbf{\textit{#1}}}}
\newcommand{\ErrorTok}[1]{\textcolor[rgb]{0.64,0.00,0.00}{\textbf{#1}}}
\newcommand{\ExtensionTok}[1]{#1}
\newcommand{\FloatTok}[1]{\textcolor[rgb]{0.00,0.00,0.81}{#1}}
\newcommand{\FunctionTok}[1]{\textcolor[rgb]{0.13,0.29,0.53}{\textbf{#1}}}
\newcommand{\ImportTok}[1]{#1}
\newcommand{\InformationTok}[1]{\textcolor[rgb]{0.56,0.35,0.01}{\textbf{\textit{#1}}}}
\newcommand{\KeywordTok}[1]{\textcolor[rgb]{0.13,0.29,0.53}{\textbf{#1}}}
\newcommand{\NormalTok}[1]{#1}
\newcommand{\OperatorTok}[1]{\textcolor[rgb]{0.81,0.36,0.00}{\textbf{#1}}}
\newcommand{\OtherTok}[1]{\textcolor[rgb]{0.56,0.35,0.01}{#1}}
\newcommand{\PreprocessorTok}[1]{\textcolor[rgb]{0.56,0.35,0.01}{\textit{#1}}}
\newcommand{\RegionMarkerTok}[1]{#1}
\newcommand{\SpecialCharTok}[1]{\textcolor[rgb]{0.81,0.36,0.00}{\textbf{#1}}}
\newcommand{\SpecialStringTok}[1]{\textcolor[rgb]{0.31,0.60,0.02}{#1}}
\newcommand{\StringTok}[1]{\textcolor[rgb]{0.31,0.60,0.02}{#1}}
\newcommand{\VariableTok}[1]{\textcolor[rgb]{0.00,0.00,0.00}{#1}}
\newcommand{\VerbatimStringTok}[1]{\textcolor[rgb]{0.31,0.60,0.02}{#1}}
\newcommand{\WarningTok}[1]{\textcolor[rgb]{0.56,0.35,0.01}{\textbf{\textit{#1}}}}
\usepackage{graphicx}
\makeatletter
\def\maxwidth{\ifdim\Gin@nat@width>\linewidth\linewidth\else\Gin@nat@width\fi}
\def\maxheight{\ifdim\Gin@nat@height>\textheight\textheight\else\Gin@nat@height\fi}
\makeatother
% Scale images if necessary, so that they will not overflow the page
% margins by default, and it is still possible to overwrite the defaults
% using explicit options in \includegraphics[width, height, ...]{}
\setkeys{Gin}{width=\maxwidth,height=\maxheight,keepaspectratio}
% Set default figure placement to htbp
\makeatletter
\def\fps@figure{htbp}
\makeatother
\setlength{\emergencystretch}{3em} % prevent overfull lines
\providecommand{\tightlist}{%
  \setlength{\itemsep}{0pt}\setlength{\parskip}{0pt}}
\setcounter{secnumdepth}{-\maxdimen} % remove section numbering
\ifLuaTeX
  \usepackage{selnolig}  % disable illegal ligatures
\fi
\IfFileExists{bookmark.sty}{\usepackage{bookmark}}{\usepackage{hyperref}}
\IfFileExists{xurl.sty}{\usepackage{xurl}}{} % add URL line breaks if available
\urlstyle{same}
\hypersetup{
  pdftitle={Student's Behavior Report},
  pdfauthor={Brenda Ramirez \& Priscila Pena},
  hidelinks,
  pdfcreator={LaTeX via pandoc}}

\title{Student's Behavior Report}
\author{Brenda Ramirez \& Priscila Pena}
\date{DATA 2401}

\begin{document}
\frame{\titlepage}

\begin{frame}{Introduction}
\protect\hypertarget{introduction}{}
\href{https://www.kaggle.com/datasets/gunapro/student-behavior/data}{Student
Behavior Data Link}

Do students from a different country have a similar college experience
as you? An individual's college experience tends to be incredibly unique
and dependent on many personal factors. However, are there some
similarities or differences in student behavior from other countries
such as India? The dataset \emph{Student Behavior, Understanding
Students: Behaviors, Preferences, and Expectations} is a collection of
insight into the behaviors and characteristics of different college
students from India.~

We didn't have much information on the data origins but we were still
very interested to see if these students had a similar college
experience as us and if there were any interesting patterns among them.
We used values such as stress level, gender, social media engagement,
financial status, hobbies, college marks, degree satisfaction,
willingness to pursue a career, salary expectation, and whether the
student had a part-time job or not. After filtering the raw data, we
dove into finding the relationships between stress levels and gender,
the correlation between gender and academic marks and patterns among
financial situations and the number of students who had part-time jobs
while in university among other questions.~
\end{frame}

\begin{frame}{Questions}
\protect\hypertarget{questions}{}
\begin{itemize}
\item
  What are the stress levels of students by gender?
\item
  Is there a relationship between social media usage and their stress
  level?
\item
  Do students with a low financial status tend to have higher stress
  levels?
\item
  Is there a relationship between the type of hobby a student has and
  their college mark?
\item
  Is there a relationship between students that like their degree and
  their willingness to pursue a career out of it?
\item
  What hobby is the most popular among students?
\item
  What gender tends to have higher academic marks?
\item
  Do students who perceive themselves to be in a low financial situation
  tend to have a part time job?
\end{itemize}
\end{frame}

\begin{frame}[fragile]{Packages Used}
\protect\hypertarget{packages-used}{}
Our project only required us to use 3 packages: \emph{tidyverse} for
cleaning and presenting data, \emph{dplyr} for data manipulation, and
\emph{ggplot} for data visualization.

\begin{Shaded}
\begin{Highlighting}[]
\FunctionTok{library}\NormalTok{(tidyverse) }\CommentTok{\#aids in cleaning and presenting data}
\FunctionTok{library}\NormalTok{(dplyr) }\CommentTok{\#data manipulation}
\FunctionTok{library}\NormalTok{(ggplot2) }\CommentTok{\#data visualization}
\end{Highlighting}
\end{Shaded}
\end{frame}

\begin{frame}[fragile]{Unfiltered Data}
\protect\hypertarget{unfiltered-data}{}
This data set from Kaggle "Student Behavior" was collected from students
through a survey in Chennai, India. At first glance, you can see that
the data we are working with is mainly categorical with only a few
exceptions to this. We can tell that we will be working with discrete
data.~

\begin{verbatim}
## Rows: 235
## Columns: 19
## $ Certification.Course                                 <chr> "No", "No", "Yes"~
## $ Gender                                               <chr> "Male", "Female",~
## $ Department                                           <chr> "BCA", "BCA", "BC~
## $ Height.CM.                                           <dbl> 100, 90, 159, 147~
## $ Weight.KG.                                           <dbl> 58, 40, 78, 20, 5~
## $ X10th.Mark                                           <dbl> 79.0, 70.0, 69.5,~
## $ X12th.Mark                                           <dbl> 65.00, 80.00, 61.~
## $ college.mark                                         <dbl> 80, 70, 55, 58, 3~
## $ hobbies                                              <chr> "Video Games", "C~
## $ daily.studing.time                                   <chr> "0 - 30 minute", ~
## $ prefer.to.study.in                                   <chr> "Morning", "Morni~
## $ salary.expectation                                   <int> 40000, 15000, 130~
## $ Do.you.like.your.degree.                             <chr> "No", "Yes", "Yes~
## $ willingness.to.pursue.a.career.based.on.their.degree <chr> "50%", "75%", "50~
## $ social.medai...video                                 <chr> "1.30 - 2 hour", ~
## $ Travelling.Time                                      <chr> "30 - 60 minutes"~
## $ Stress.Level                                         <chr> "Bad", "Bad", "Aw~
## $ Financial.Status                                     <chr> "Bad", "Bad", "Ba~
## $ part.time.job                                        <chr> "No", "No", "No",~
\end{verbatim}
\end{frame}

\begin{frame}{Data Key Terms}
\protect\hypertarget{data-key-terms}{}
Stress Level : perceived stress level of the student

Gender

Social Media \& Video : engagement with these platforms

Financial Status : economic background or current financial status of
the student

Hobbies : interests of the students

College Mark : grades obtained in their college or university * since
its from another country their grade scale is different

Do you like your degree : opinion on whether they like their degree or
not

Willingness to pursue a career based on their degree : a percentage of
that likeliness

Salary expectation : future salary expectations

Part-time job : whether or not a student is engaged in one
\end{frame}

\begin{frame}[fragile]{Difference in gender data}
\protect\hypertarget{difference-in-gender-data}{}
When we first tackled this question we noticed a large gap in our data
between the genders. There were more males than females by almost double
the amount. We knew this would have a significant effect on our findings
so we decided to take a random sample of males in this data. Our first
task was to visualize the ratio between men and women surveyed. We first
turned the counts of the original data for each gender to find 79 female
and 156 male students were surveyed. Then, we turned these numbers into
percentages to show the overall distribution. Finally, we used a bar
chart to contrast the percentages of each gender category and found the
male data almost doubled the female. 34\% of students were female and
66\% were male.

\begin{verbatim}
##   Gender   n
## 1 Female  79
## 2   Male 156
\end{verbatim}

\includegraphics{ProjectReport_files/figure-beamer/unnamed-chunk-3-1.pdf}
\end{frame}

\begin{frame}[fragile]{Random Male Sample}
\protect\hypertarget{random-male-sample}{}
Since we found that our data was skewed due to the male data almost
doubling the amount of female data, we created a random male sample
proportional to that of the women. To do this, we created a new variable
"male sample" in which we randomly sampled 79 male students to equal the
amount of female data. This way our male-to-female visualizations would
be proportional and this sample fixes the issue of a skewed dataset.

\begin{verbatim}
##    Gender Stress.Level
## 1    Male         Good
## 2    Male         Good
## 3    Male         Good
## 4    Male        Awful
## 5    Male         Good
## 6    Male     fabulous
## 7    Male         Good
## 8    Male        Awful
## 9    Male          Bad
## 10   Male          Bad
## 11   Male     fabulous
## 12   Male         Good
## 13   Male         Good
## 14   Male         Good
## 15   Male          Bad
## 16   Male          Bad
## 17   Male         Good
## 18   Male         Good
## 19   Male         Good
## 20   Male          Bad
## 21   Male          Bad
## 22   Male         Good
## 23   Male          Bad
## 24   Male         Good
## 25   Male        Awful
## 26   Male          Bad
## 27   Male          Bad
## 28   Male         Good
## 29   Male          Bad
## 30   Male        Awful
## 31   Male        Awful
## 32   Male         Good
## 33   Male         Good
## 34   Male         Good
## 35   Male         Good
## 36   Male          Bad
## 37   Male          Bad
## 38   Male         Good
## 39   Male         Good
## 40   Male     fabulous
## 41   Male         Good
## 42   Male          Bad
## 43   Male         Good
## 44   Male        Awful
## 45   Male          Bad
## 46   Male         Good
## 47   Male     fabulous
## 48   Male         Good
## 49   Male         Good
## 50   Male         Good
## 51   Male         Good
## 52   Male         Good
## 53   Male         Good
## 54   Male         Good
## 55   Male         Good
## 56   Male         Good
## 57   Male         Good
## 58   Male        Awful
## 59   Male         Good
## 60   Male         Good
## 61   Male         Good
## 62   Male         Good
## 63   Male     fabulous
## 64   Male          Bad
## 65   Male         Good
## 66   Male         Good
## 67   Male         Good
## 68   Male          Bad
## 69   Male          Bad
## 70   Male         Good
## 71   Male          Bad
## 72   Male         Good
## 73   Male        Awful
## 74   Male          Bad
## 75   Male         Good
## 76   Male          Bad
## 77   Male         Good
## 78   Male          Bad
## 79   Male          Bad
\end{verbatim}
\end{frame}

\begin{frame}{What are the stress levels of students by gender?}
\protect\hypertarget{what-are-the-stress-levels-of-students-by-gender}{}
To visualize the stress levels by gender, we created a new variable to
combine data from 2 sources, the original data and the male sample we
created. We then mutated the gender column so that it could use the male
sample instead of the original. Finally, we created a grouped bar graph
to show the comparison between the 2 genders and the different stress
levels.

\includegraphics{ProjectReport_files/figure-beamer/unnamed-chunk-5-1.pdf}

The stress levels between students are very close to one another. There
are however slight differences such as more men feel they have awful and
fabulous stress levels. Additionally, more female students had both bad
and good stress levels than men. This data doesn't have much of a
pattern so we can't say that stress level is influenced by gender based
on this data. Overall, based on these findings, students of both genders
do not suffer from high-stress levels.
\end{frame}

\begin{frame}{Is there a relationship between social media usage and
their stress level?}
\protect\hypertarget{is-there-a-relationship-between-social-media-usage-and-their-stress-level}{}
To find this relationship, we created 2 new variables based on social
media and student stress levels to organize the order of the values of
each category. Next, we used a facet wrap to create separate facets for
each category of social media usage. This made it easier to compare and
contrast stress level distributions within each social media usage
category.

\includegraphics{ProjectReport_files/figure-beamer/unnamed-chunk-6-1.pdf}

We were expecting to see lower stress levels from students who used the
least amount of social media and higher stress levels if they spent more
than 2 hours on it. We were very surprised to find that most students
with low stress levels spend around 1-30 minutes, 30-60 minutes, or
1-1.30 hours on social media. These students did spend lower periods on
social media, but students with the worst stress levels, seem to not
have used social media as much as we would've thought.
\end{frame}

\begin{frame}{Do students with a low financial status have higher stress
levels?}
\protect\hypertarget{do-students-with-a-low-financial-status-have-higher-stress-levels}{}
We once again used a facet wrap to find the relationship between a
student's financial status and their stress levels. We were expecting to
find higher stress levels for students who had a lower financial status.
Stress levels seem to be similarly bad regardless if a student has a
good or bad financial status. However, we found that these students'
stress levels are overall low despite their financial status.

\includegraphics{ProjectReport_files/figure-beamer/unnamed-chunk-7-1.pdf}
\end{frame}

\begin{frame}{Is there a relationship between the type of hobby a
student has and their college mark?}
\protect\hypertarget{is-there-a-relationship-between-the-type-of-hobby-a-student-has-and-their-college-mark}{}
For this relationship, we used a box plot to summarize the distribution
of college marks based on hobby preference. There were a couple of
outliers in our data, for example 3 very low marks from students who
prefer to read books and overall lower marks from students who play
sports. However, the range of college marks stays between 50-100\% no
matter what type of hobby the student prefers.

\includegraphics{ProjectReport_files/figure-beamer/unnamed-chunk-8-1.pdf}
\end{frame}

\begin{frame}{Is there a relationship between students that like their
degree and their willingness to pursue a career out of it?}
\protect\hypertarget{is-there-a-relationship-between-students-that-like-their-degree-and-their-willingness-to-pursue-a-career-out-of-it}{}
We used a facet wrap again here. We based it on whether students liked
their degree or not along with the percentage of willingness a student
has to pursue a career based on that degree. We found that the majority
of students who liked their degree were more willing to pursue a career
based on their degree. However, it was interesting to find that there
are a couple of students who are still pursuing a particular career
despite not liking it.

\includegraphics{ProjectReport_files/figure-beamer/unnamed-chunk-9-1.pdf}
\end{frame}

\begin{frame}{What hobby is the most popular among students?}
\protect\hypertarget{what-hobby-is-the-most-popular-among-students}{}
We chose a bar graph to show which hobby is the most popular among
students. We created a new variable to show the percentage count of each
hobby and used geom\_text to add the labels at the top of the bars. This
graph is also organized in descending percentage order so the most
popular hobby is easier to spot. 15\% of students prefer to read books,
another 15\% like video games, 33\% chose cinema and 36\% chose
sports.~We can assume that sports were the most popular in this data
because of the abundance of males in the data

\includegraphics{ProjectReport_files/figure-beamer/unnamed-chunk-10-1.pdf}
\end{frame}

\begin{frame}{What gender tends to have higher academic marks?}
\protect\hypertarget{what-gender-tends-to-have-higher-academic-marks}{}
To find which gender has higher academic marks, we used a violin plot to
see the shape and density of the distribution. We then flipped the
graphs so that it was easier to compare them.Most marks range around the
80\textquotesingle s. We found that women tend to have higher academic
marks with a higher minimum and higher maximum. Men tend to have a lower
max and a lower minimum that goes all the way down to 0\%. Their marks
are also below average ranging in the 70 \%s.

\includegraphics{ProjectReport_files/figure-beamer/unnamed-chunk-11-1.pdf}
\end{frame}

\begin{frame}{Do students who perceive themselves to be in a low
financial situation tend to have a part time job?}
\protect\hypertarget{do-students-who-perceive-themselves-to-be-in-a-low-financial-situation-tend-to-have-a-part-time-job}{}
We also approached this question with a facet wrap.

\includegraphics{ProjectReport_files/figure-beamer/unnamed-chunk-12-1.pdf}

You would think that more students would have part-time jobs if their
financial situation is not very good, but in this data, we see that the
majority of students do not have a part-time job. The amount of students
that work is about the same regardless if their financial status is good
or bad. Even though they have a bad financial situation, because
they\textquotesingle re students, they don't seem to start working right
away.
\end{frame}

\begin{frame}{Summary}
\protect\hypertarget{summary}{}
Based on our data manipulation and visualization in this data set these
are some of the things we were able to conclude. we found a positive
relationship between the number of students who like their degrees and
their willingness to pursue a career out of it. We also found that women
tend to have higher marks than men. A student's stress levels may not be
directly influenced by their gender, their social media use, or their
financial status. The type of hobby a student chooses also did not
affect the type of academic marks they received. Because they are still
students, their financial status does not define if they have a
part-time job or not. A lot of our original hypotheses were proven wrong
with our data findings. We concluded that since students come from many
distinct backgrounds and live in different situations, you can't assume
that students' college experiences are going to be the same. You can't
fit college students into a one-size-fits-all mold.~
\end{frame}

\end{document}
